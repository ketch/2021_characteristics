\documentclass[11pt]{article}

\usepackage[T1]{fontenc}
\usepackage[utf8]{inputenc}
%\usepackage[ngerman,english]{babel}
\usepackage[hmargin=0.7in, vmargin=0.8in]{geometry}
\usepackage[babel,german=quotes]{csquotes}
\usepackage{subfig}
\usepackage{graphicx}
\usepackage[colorlinks=true,citecolor=blue,urlcolor=blue]{hyperref}
\usepackage{caption}
%\usepackage{subcaption}
\usepackage{amsmath}          % writing mathematical formulas
\usepackage{amsthm}           % writing mathematical theorems 
\usepackage{amssymb}          % writing mathematical symbols
\usepackage{bm, amsfonts}     % writing bold mathematical symbols
\usepackage{xcolor}
\usepackage{fixmath}
\usepackage{tikz}
\usepackage{bm}
\usepackage{makecell}
\usepackage{mathdots}
\usepackage{algpseudocode}
\usepackage{algorithm}

\begin{document}

\title{Practical calculations for the path-integral approach to
  solving the wave equation with continuously-varying coefficients}
  
\author{David I. Ketcheson}

\maketitle

To make things simple, we take
\begin{align}
x_0 & = 0 & x_+ & = 1 \\
c(x) & = 1 & Z(x) & = e^{2x}.
\end{align}
Then $r(x)=1$, so all the integrals we need to compute are simply volumes
of certain high-dimensional polyhedra.  These can be computed quickly using 
Mathematica (for reasonable orders).  It is also convenient to reparameterize
using $s=(t-1)/2$ for the transmission integrals (except $T_0$) and $s=t/2$
for the reflection integrals.

Some of the results are (we omit the zero case for s<0):
\begin{align}
T_0(t)/C_G & = 
    \begin{cases} 0 & t<1  \\ 1 & t \ge 1.
    \end{cases} \\
T_2(s)/C_G & =
    \begin{cases} \frac{1}{2}s(2-s) & 0<s<1  \\
                    \frac{1}{2} & s \ge 1.
    \end{cases} \\
T_4(s)/C_G & =
    \begin{cases} \frac{1}{12}s^2(3-s^2) & 0<s<1  \\
                    \frac{1}{24}\left(5-(s-2)^4\right) & 1 < s < 2 \\
                    \frac{5}{24} & s \ge 2.
    \end{cases} \\
T_6(s)/C_G & = 
\begin{cases}
 \frac{1}{720} \left(-5 s^6-6 s^5+15 s^4+20 s^3\right) & 0<s<1 \\
 \frac{1}{720} \left(-4 s^6+24 s^5-15 s^4-180 s^3+465 s^2-366 s+100\right) & 1<s<2 \\
 \frac{1}{720} \left(61-(s-3)^6\right) & 2<s\leq 3 \\
 \frac{61}{720} & s>3 \\
\end{cases}
\end{align}
We observe several things, some of which are easily proved:
\begin{enumerate}
    \item Each function $T_{2m}(s)/C_G$ is a piecewise polynomial of degree $2m$
            with breakpoints at the integer values of $s$ and is constant for
            $s$ outside $[0,m]$.
    \item The value at $s=m$ is given by $A_{2m}/(2m!)$, where $A_j$ are the Euler
            zigzag numbers.
    \item The value on the interval $m-1<s<m$ is given by
        $$
            \frac{1}{(2m)!}\left(A_{2m} - (s-m)^{2m}\right).
        $$
    \item The function is continuous.  Moreover, at $s=m$ the first $2m-1$ derivatives are
        continuous and the value of the $2m$th derivative is $-1$.  This completely
        determines the last piecewise segment.
    \item The function value at $s=m-1$ is $A_{2m-1}/(2m!)$.  Thus these functions
        change very little between $s=m-1$ and $s=m$.
    \item At $s=m-1$, the function and first $2m-2$ derivatives are continuous.  At $s=m-2$,
            the function and first $2m-3$ derivatives are continuous.  At $s=m-j$, the
            function and first $2m-j-1$ derivatives are continuous until at $s=0$
            the function and first $m-1$ derivatives are continuous.
    \item At $s=0$, the $m$th derivative is equal to $1/m!$.
\end{enumerate}
Evidently the degree of continuity at each node has to do with where in the
chain of integrals we need to split things up into cases.
The conditions just enumerated are enough to completely determine $T_0$ through $T_4$,
but not quite $T_6$, and the constraint deficit will grow with $m$.  It would be nice
to figure out how to compute these for arbitrary $m$, although from a practical point
of view going up to $T_8$ may be more than enough.  My main proposed technique for
getting further is to notice patterns and conjecture the answers.

Things to do:
\begin{itemize}
    \item Confirm these results (and corresponding ones for reflection) by comparing with
            PyClaw simulations.
    \item Figure out whether these formulas can easily be adapted to work for arbitrary
            variation in $Z(x)$ (possibly restricted to monotone functions).
            That would make this a powerful computational tool (and much easier to use).
    \item Figure out the best way to use this for more complicated (e.g. non-monotone,
            periodic) impedance profiles.  The Green's functions are given by the 
            derivatives of the functions above, so one approach is to convolve those.
            Can we somehow get dispersion and Airy functions out of this?

\end{itemize}


%\bibliographystyle{plain}
%\bibliography{refs}
\end{document}
